%%%%%%%%%%%%%%%%%%%%%%%%%%%%%%%%%%%%%%%%%
% Medium Length Professional CV
% LaTeX Template
% Version 2.0 (8/5/13)
%
% This template has been downloaded from:
% http://www.LaTeXTemplates.com
%
% Original author:
% Rishi Shah 
%
% Important note:
% This template requires the resume.cls file to be in the same directory as the
% .tex file. The resume.cls file provides the resume style used for structuring the
% document.
%
%%%%%%%%%%%%%%%%%%%%%%%%%%%%%%%%%%%%%%%%%

%----------------------------------------------------------------------------------------
%	PACKAGES AND OTHER DOCUMENT CONFIGURATIONS
%----------------------------------------------------------------------------------------

\documentclass{resume} % Use the custom resume.cls style
\usepackage{enumitem}
\usepackage[left=0.6in,top=0.6in,right=0.6in,bottom=0.6in]{geometry} % Document margins
\newcommand{\tab}[1]{\hspace{.2667\textwidth}\rlap{#1}}
\newcommand{\itab}[1]{\hspace{0em}\rlap{#1}}
\name{Sean Chapman} % Your name
%\address{123 Pleasant Lane \\ City, State 12345} % Your secondary addess (optional)
\address{(650) 353-6175 \\ seankchapman@gmail.com \\ github.com/seankchapman \\ seankchapman.net \\ San Mateo, CA} % Your phone number and email

\begin{document}

%----------------------------------------------------------------------------------------
%	EDUCATION SECTION
%----------------------------------------------------------------------------------------

\begin{rSection}{Education}

{\bf University of Maryland, College Park} {\em | College Park, MD} \hfill {\em August 2017 - May 2021} 
\\ B.S. Computer Science
\\ Minor in Astronomy

{\bf Relevant Coursework}
\\ Computer Systems / Algorithms / Advanced Data Structures / Data Science / Full-Stack Web Development / Computer Networks + Security / Machine Learning / iOS Development / Concurrent and Distributed Computing

%Minor in Linguistics \smallskip \\
%Member of Eta Kappa Nu \\
%Member of Upsilon Pi Epsilon \\


\end{rSection}

\begin{rSection}{Skills}

\begin{tabular}{ @{} >{\bfseries}l @{\hspace{6ex}} l }
Proficient in\ & Java, Python, C, Git, UNIX, HTML, CSS, Javascript \\
Familiar with & Node.js, MongoDB, SQL, C\#, Swift, Hadoop, MapReduce \\
\end{tabular}

\end{rSection}

%----------------------------------------------------------------------------------------
%	WORK EXPERIENCE SECTION
%----------------------------------------------------------------------------------------

\begin{rSection}{Experience}
\begin{rSubsection}{{\bf Software Research Intern (VR)} | {\em NTU IoX Center (Taipei, TW)}}{\em July 2019 - August 2019}{}{}
\begin{itemize}
\item Developed VR demos in Unity and C\# to analyze novel VR haptics systems. \vspace{-0.6em}
\item Wired and programmed arduinos for a prototype VR haptic system called the GuideBand.\vspace{-0.6em}
\item Co-authored a paper on providing multi-level non-uniform feedback on the feet in VR (FrictShoes).

\end{itemize}
\end{rSubsection}
\begin{rSubsection}{{\bf Software Engineer Intern} | {\em Wonplanet.com (Palo Alto, CA)}}{\em June 2018 - August 2018}{}{}
\begin{itemize}
\item Worked on the full-stack implementation of premium subscription functionality.\vspace{-0.6em}
\item Utilized the Braintree API along with Scala, HTML, CSS, and the Play Framework to implement payment integration.\vspace{-0.6em}
\item Updated User SQL schemas to account for varying subscription plans.\vspace{-0.6em}
\end{itemize}
\end{rSubsection}
\end{rSection}

% Developed VR demos in Unity and C\# to analyze novel VR haptics systems. Wired and programmed arduinos for prototypes
%--------------------------------------------------------------------------------
%    Projects And Seminars
%-----------------------------------------------------------------------------------------------
\begin{rSection}{Projects}

\begin{rSubsection}{\bf Akka Resource Manager \hspace{0.1em} \it (Java)}{\em April 2021}{}{}
\begin{itemize}
    \item Developed a Java program that utilizes the Akka Framework to create a distributed resource management system.\vspace{-0.6em}
    \item Allows users to access and modify files across a distributed network of computers.
\end{itemize}
\end{rSubsection}

\begin{rSubsection}{\bf Multi-threaded Maze Solver \hspace{0.1em} \it (Java)}{\em March 2021}{}{}
\begin{itemize}
    \item Utilized Java thread-pools and task schedulers to efficiently calculate solutions to extremely large mazes.\vspace{-0.6em}
    \item Achieved solution speed in top 15\% of class.
\end{itemize}
\end{rSubsection}

\begin{rSubsection}{\bf SDSS Classification \hspace{0.1em} \it (Python)}{\em June 2020}{}{}
\begin{itemize}
    \item Performed an exploratory analysis of data from NASA's Sloan Digital Sky Survey. \vspace{-0.6em}
    \item Utilized Python's data science stack to generate interesting visualizations. \vspace{-0.6em}
    \item Created classifiers to determine whether an object is a star, galaxy, or quasar with up to 98\% accuracy.
\end{itemize}
\end{rSubsection}


\begin{rSubsection}{\bf Terp Food Reviewer \hspace{0.1em} \it (Javascript)}{\em May 2019}{}{}
\begin{itemize}
    \item Built and deployed a live-updating hub of local restaurant reviews for students at the University of Maryland.\vspace{-0.6em}
    \item Utilized Node.js, MongoDB for the back-end. Express, Handlebars, and Javascript for the front-end.
\end{itemize}
\end{rSubsection}


\end{rSection}

%----------------------------------------------------------------------------------------
% Extra Curricular
%----------------------------------------------------------------------------------------

\begin{rSection}{Publications} \itemsep -3pt
\begin{rSubsection}{\bf FrictShoes: Providing Multilevel Nonuniform Friction Feedback on Shoes in VR (TAICHI 2021)}{\em July 2021}{}{}
\begin{itemize}
    \item Co-authored a research paper while working as a research intern at the NTU IoX Center in the summer of 2019.\vspace{-0.6em}
    \item Proposed a wearable device, FrictShoes, to provide multilevel nonuniform friction feedback on feet.
\end{itemize}
\end{rSubsection}
\end{rSection}

\begin{rSection}{Extracurriculars} \itemsep -3pt
\begin{rSubsection}{\bf Hackital 2018}{\em December 2018}{}{}
\begin{itemize}
    \item One of three finalists at a cryptocurrency-themed hackathon hosted by students at George Washington University. \vspace{-0.6em}
    \item Developed a trivia game that runs on the Ethereum blockchain platform.
\end{itemize}
\end{rSubsection}

\begin{rSubsection}{{\bf Palo Alto High School Robotics Team}}{\em October 2015 - June 2016}{}{}
\begin{itemize}
    \item Updated information and UI features for the team's website with HTML/CSS/Javascript.
\end{itemize}
\end{rSubsection}

\end{rSection}



%----------------------------------------------------------------------------------------
%	TECHNICAL STRENGTHS SECTION
%----------------------------------------------------------------------------------------
\end{document}
